\documentclass[12pt,a4paper,oneside]{report}
\usepackage{amsmath}
\usepackage[utf8]{inputenc}
\usepackage[russian]{babel}
\usepackage[left=2cm,right=1.5cm, top=2cm,bottom=2cm,bindingoffset=0cm]{geometry}

\title{Лабораторная работа №4\\по курсу\\Методы вычислений \\<<Минимизация функций двух переменных при наличии ограничений.>>}
\author{Фроловский Алексей Вадимович}
\date{группа ИУ7-17 \\ Вариант № 14\\5 декабря 2012 года}

\begin{document}
\maketitle
\section*{Цель работы}

Написать программы нахождения минимума функции при наличии онраничений, реализующие:
\begin{enumerate}
\item Метод штрафных функций;
\item Метод барьерных функций.
\end{enumerate}

Найти точку минимума квадратичной функции
\begin{equation}
\label{f1}
z = 4x_{1}x_{2}+7x^{2}_{1}+4x^{2}_{2}+6\sqrt{5}x_{1}-12\sqrt{5}x_{2}+51
\end{equation}
при наличии следующих ограничений:
\begin{equation}
\left\{  
 	\begin{array}{rcl}  
	4x_{1} + 5x_{2} & \leq & 0 \\  
           x_{1} + x^{2}_{2} & \leq & 0 \\  
           -6x_{1} - x_{2} - 26 & \leq & 0 \\  
           \end{array}   
\right.
\end{equation}
с использованием указанных выше методов.

\section*{Ход работы}

Поскольку конечное перечение выпуклых множеств является выпуклым множеством, то образованное ограничениями множество является выпуклым.

Покажем. что функция ~(\ref{f1}) является выпуклой с использованием критерия Сильвестра.

\begin{equation}
f^{\prime\prime}_{xx} =  \left( \begin{array}{cc}
	14 & 4 \\
 	  4 & 8 \end{array} \right)
\end{equation}

Как можно увидеть данная матрица является положительно определенной и, следовательно, функция ~(\ref{f1}) является выпуклой.

Результаты вычисления минимума функции разными методами представлены  таблице.
\begin{center}
\begin{tabular}{||c|c|c|c||}
\hline
Метод		&	$x^{*}$	&	$f(x^{*})$  &	количество вычислений функции \\
\hline
Penalty functions & (-1.9572,  1.3991) & 10.8907 & 3439 \\\hline 
 
\hline
Barier functions & (-1.9566,  1.3988) & 10.8930 & 2441 \\\hline 
 
\hline
\end{tabular}
\end{center}

Проверим выполнение условий Куно-Такера для полученных решений.

\begin{equation}
\left\{  
 	\begin{array}{rcl}  
           \dfrac{\partial{f}}{\partial{x_{1}}} + \sum\limits_{i=1}^3\lambda_{i}\dfrac{\partial{g_{i}}}{\partial{x_{1}}} & = & 4x_{2} + 14x_{1} + 6\sqrt{5} + \lambda_{1}(1) + \lambda_{2}(1) + \lambda_{3}(-6)  = 0\\  
           \dfrac{\partial{f}}{\partial{x_{1}}} +  \sum\limits_{i=1}^3\lambda_{i}\dfrac{\partial{g_{i}}}{\partial{x_{2}}} & = &4x_{1} + 8x_{2} - 12\sqrt{5} + \lambda_{1}(5) + \lambda_{2}(2x_{2}) + \lambda_{3}(-1)  = 0\\
	\lambda_{1}g_{1}(x) & = & \lambda_{1}(4x_{1} + 5x_{2})  =  0 \\  
	\lambda_{2}g_{2}(x) & = & \lambda_{2}(x_{1} + x^{2}_{2})  =  0 \\  
	\lambda_{3}g_{3}(x) & = & \lambda_{3}(-6x_{1} - x_{2} - 26)  =  0 \\  
	g_{1}(x) & = & 4x_{1} + 5x_{2}  \leq  0 \\  
           g_{2}(x) & = & x_{1} + x^{2}_{2}  \leq  0 \\  
           g_{3}(x) & = & -6x_{1} - x_{2} - 26 \leq  0 \\  
           \end{array}   
\right.
\end{equation}

Заметим, что ограничения $g_{1}(x)$ и $g_{3}(x)$ являются неактивными, поэтому $\lambda_{1} = 0$ и
$\lambda_{3} = 0$. 

В итоге получаем 2 уравнения:

\begin{equation}
\begin{array}{rcl}  
 4x_{2} + 14x_{1} + 6\sqrt{5} +  \lambda_{2}(1)  = 0\\  
 4x_{1} + 8x_{2} - 12\sqrt{5} +  \lambda_{2}(2x_{2}) = 0\\
\end{array}   
\end{equation}

получаемые значения $\lambda_{2}$ из которых должны быть примерно равны.

Для решения полученного методом штрафных функций имеем:
\begin{equation}
\begin{array}{rcl}  
\lambda_{2}  =  8.3884\\  
\lambda_{2} =  8.3870\\
\end{array}   
\end{equation}


Для решения полученного методом барьерных функций имеем:

\begin{equation}
\begin{array}{rcl}  
\lambda_{2}  =  8.3812\\  
\lambda_{2} =  8.3890\\
\end{array}   
\end{equation}

Исходя из полученных значений $\lambda_{2}$ можно сделать вывод, что условия Куно-Такера выполнены с точность до сотых и полученные точки являются решениями поставленной выше задачи.

\end{document}
