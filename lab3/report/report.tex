\documentclass[12pt,a4paper,oneside]{report}
\usepackage{amsmath}
\usepackage[utf8]{inputenc}
\usepackage[russian]{babel}
\usepackage[left=2cm,right=1.5cm, top=2cm,bottom=2cm,bindingoffset=0cm]{geometry}

\title{Лабораторная работа №3\\по курсу\\Методы вычислений \\<<Безусловная минимизация функций двух перемен-ных. Методы, использующие производные.>>}
\author{Фроловский Алексей Вадимович}
\date{группа ИУ7-17 \\ Вариант № 14\\23 ноября 2012 года}

\begin{document}
\maketitle
\section*{Цель работы}
Написать программы нахождения минимума, реализующие:
\begin{enumerate}
\item Метод сопряженных градиентов;
\item Метод Ньютона с конечно-разностной аппроксимацией производных;
\item Метод Девидона–Флетчера–Пауэла (ДФП).
\end{enumerate}

Найти точку минимума квадратичной функции
\begin{equation}
\label{f1}
z = 4x_{1}x_{2}+7x^{2}_{1}+4x^{2}_{2}+6\sqrt{5}x_{1}-12\sqrt{5}x_{2}+51
\end{equation}
с использованием указанных выше методов, а также используя возможности Optimization Toolbox Matlab. В качестве стартовой взять точку~$(0;-\sqrt{5})$. 

Для функции 
\begin{equation}
\label{f2}
z = x^{3}_{2}+2x_{2}x_{1} + \frac{1}{\sqrt{x_{1}x_{2}}} + x_{1}
\end{equation}
с помощью созданнх программ найти локальный минимум, ближайший к стартовой точке~$(3, 3)$.

Критерий окончания выбрать так, чтобы координаты вычисленного приближения к
точке минимума функции содержали три верные значащие цифры.

\section*{Ход работы}
Найдем минимум функции~(\ref{f1}) теоритическим способом. Cocтавим и решим систему уравнений (\ref{eqsys})
\begin{equation}
\label{eqsys}
\left\{  
 	\begin{array}{rcl}  
           \dfrac{\partial{f}}{\partial{x_{1}}} & = & 0 \\  
           \dfrac{\partial{f}}{\partial{x_{2}}} & = & 0 \\  
           \end{array}   
 \right.
\end{equation}
или
\begin{equation}
\label{eqsysext}
\left\{  
 	\begin{array}{rcl}  
           4x_{2} + 14x_{1} + 6\sqrt{5} & = & 0 \\  
           4x_{1} + 8x_{2} - 12\sqrt{5} & = & 0 \\  
           \end{array}   
 \right.
\end{equation}
Получим $x_{1} = -\sqrt{5}, x_{2} = 2\sqrt{5}$. Подставим полученные значения  в (\ref{f1}):
\begin{equation}
f(-\sqrt{5},2\sqrt{5}) = -24
\end{equation}
Таким образом, минимальное значение функции -24 достигается в точке (-2.2362,  4.4719).

Покажем. что функция ~(\ref{f1}) является выпуклой.

\begin{equation}
f^{\prime\prime}_{xx} =  \left( \begin{array}{cc}
	14 & 4 \\
 	  4 & 8 \end{array} \right)
\end{equation}

Как можно увидеть данная матрица является положительно определенной и, следовательно, функция ~(\ref{f1}) является выпуклой.


Доопределим функцию~(\ref{f2}) так, чтобы обеспечить нахождение требуемого локального минимума. Для этого наложим на область определения ограничения $x_{1}>0, x_{2}>0$
\section*{Результаты}
 Сравнение работы методов для квадратичной функции (\ref{f1})
\begin{center}
\begin{tabular}{||c|c|c|c||}
\hline
Метод		&	$x^{*}$	&	$f(x^{*})$  &	количество вычислений функции \\
\hline
Standart & (-2.2362,  4.4719) & -24.0000 & 85 \\\hline 
 
\hline
Conjugate gradient & (-2.2361,  4.4721) & -24.0000 & 85 \\\hline 
 
\hline
Newton method & (-2.2361,  4.4719) & -24.0000 & 53 \\\hline 
 
\hline
DFP gradient & (-2.2361,  4.4721) & -24.0000 & 21 \\\hline 
 
\hline 		
\end{tabular}
\end{center}

Для функции~(\ref{f2})
\begin{center}
\begin{tabular}{||c|c|c|c||}
\hline
Метод		&	$x^{*}$	&	$f(x^{*})$  &	количество вычислений функции \\
\hline
Standart & ( 0.4742,  0.5406) &  3.1200 & 90 \\\hline 
 
\hline
Conjugate gradient & ( 0.4742,  0.5407) &  3.1200 & 208 \\\hline 
 
\hline
Newton method & ( 0.4743,  0.5407) &  3.1200 & 166 \\\hline 
 
\hline
DFP gradient & ( 0.4743,  0.5406) &  3.1200 & 82 \\\hline 
 
\hline 		
\end{tabular}
\end{center} 

Лучшие результаты как по точности, так и по трудоемкости, показывает метод ДФП. Метод сопряженных градиентов показал значительно худшие результаты для обеих функций. С точки зрения точности поиска все три метода показали приблизительно одинаковые значения.
\end{document}
